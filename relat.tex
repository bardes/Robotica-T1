\documentclass[11pt,oneside]{article}
\usepackage[brazilian]{babel}			% Ajusta a língua para Portugês Brasileiro
\usepackage{amsmath}					% Permite escrever fórmulas matemáticas
\usepackage[utf8]{inputenc}			% Lê o arquivo fonte codificado em UTF-8
\usepackage[T1]{fontenc}				% Codifica os tipos no arquivo de saída usando T1
\usepackage[a4paper]{geometry}			% Ajusta o tamanho do papel para A4
\usepackage{titling}					% Permite fazer ajustes no título do artigo
\usepackage{fancyhdr}					% Permite ajustar cabeçalhos e rodapés
\usepackage{indentfirst}				% Ajusta a indentação do primeiro parágrafo

%% Título do artigo %%
\title{
\vspace{-3cm}	% Reduz a quantidade de espaço em branco antes do título
SSC--0712 Programação de Robôs Móveis \\
Relatório -- 1
}

%% Autores do artigo %%
\author {
Paulo Bardes (8531932)
}

%% Data %%
\date {\today}

%% Inicio do arqigo %%
\begin{document}

% Insere o título %
\maketitle
\vspace{-6mm}
\hrule

\section{Descrição}
Este relatório descreve a estratégia e implementação dos algorítimos de controle de um time composto por 3 robôs autônomos. O objetivo é coletar metade mais um dos itens verdes presentes no mapa e taze-los para a região inicial do mapa.

Como apenas 2 dos robôs são capazes de carregar itens, diferentes estratégias são necessárias para controlar cada um dos deles. Em particular, tomando vantagem do fato de que o robô que não carrega itens é mais rápido do que os outros, o que permite um mapeamento mais ágil.

\section{Implementação}

A lógica de controle dos robôs foi implementada em C++, com biblioteca \emph{libplayerc++}, e simulado no programa \emph{Stage}. O código pode ser encontrado no seguinte repositório: \texttt{github.com/bardes/Robotica-T1}.

A implementação da lógica de controle foi feita através de uma máquina de estados. A diferença entre os robôs com e sem garra é que o robô sem garra não passa pelo estado \ref{garra}.

\subsection{Prevenção de Colisão}
O algorítimo de prevenção de colisão é uma combinação entre a tática de campos potenciais e de desvio do ponto mais próximo. Para evitar discrepâncias nos dados o feixe de laser alinhado com o centro do robô é ignorado durante o cálculo dos campos potencias.

Como uma heurística para evitar colisões o robô monitora (através do laser) o ponto mais próximo e dependendo da distância para os motores para evitar contato e permitir mais tempo para as manobras de evasão.

\subsection{Navegação Ponto a Ponto}
Para navegar de um ponto ao outro o robô calcula a diferença angular entre seu ângulo atual e o angulo entre o ponto atual e o ponto de destino. Em seguida o ângulo é normalizado para o intervalo $[-\pi, \pi]$ usando-se o seguinte método: \texttt{atan2(sin(diff), cos(diff))}. Essa técnica se aproveita do fato de que as funções trigonométricas da \emph{libc} garantidamente retornam um resultado no intervalo correto.

\subsection{Mapeamento}
Para localizar os itens cada robô possuí um conjunto ordenado de pontos, definindo uma rota que deve ser percorrida. Toda a navegação é feita usando-se o algoritmo de navegação ponto a ponto, em conjunto com a prevenção de colisão, que toma prioridade em caso de conflitos.

Ao percorrer da rota quaisquer itens avistados são adicionados à uma fila comum a todos os robôs, junto com informações sobre sua localização e se este item já está delegado a algum robô. 

\subsection{Coleção de Itens}
\label{garra}
\subsubsection{Navegação Até o Item}
Todos os robôs com garras monitoram a lista em busca do item mais próximo de si. Se a lista estiver vazia todos os robôs entram em modo de mapeamento. Para navegar até o item desejado o robô primeiro navega através da rota predefinida até o ponto mais próximo do item e somente no final navega diretamente para o item.

\subsubsection{Obtenção}
Para a obtenção do item o robô usa a câmera (\emph{blobfinder}) como referência de alinhamento e aproxima-se lentamente até a distância correta do item.

\subsubsection{Navegação Até a Área Inicial}
Uma vez que o item é obtido ele deve ser trazido de volta para a área inicial do time para finalmente ser descartado. para isso o robô primeiro navega em direção ao ponto mais próximo de sua rota e em seguida segue sobre a rota até chegar na região inicial. 

\section{Resultados}
Os algorítimos de navegação de ponto a ponto e prevenção de colisão se mostraram surpreendentemente eficientes. Foram necessárias algumas iterações de testes e ajuste de parâmetros, porém ao final os resultados foram positivos até mesmo em situações desfavoráveis como obstáculos côncavos.

O algorítimo de localização e mapeamento de itens infelizmente se mostrou muito impreciso, provavelmente ajustes finos nos parâmetros do sensor \emph{blobfinder} poderiam contribuir para resultados mais confiáveis, porém devido a natureza do sensor é difícil garantir esses resultados, essencialmente fora das simulações.

Por fim a obtenção de itens não pode ser completamente implementada, apesar dos componentes básicos do algoritmo já estarem prontos (navegação ponto a ponto e rotas), devido a falta de tempo e imprecisões do mapeamento.

\end{document}
